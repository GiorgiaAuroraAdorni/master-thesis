\chapter{Model}
\label{chap:model}

\section{Imitation Learning}
\label{sec:imitlrng}
\subsection{Controllers}
\label{subsec:controllersmodel}

In a \gls{mas}, each agent can perceive the environment through sensors 
acquiring a total or partial knowledge of it. The observations extracted can be 
used by a controller, together with the current state of the agent, to determine 
actions, draw inferences and finally solve tasks. 

In an imitation learning setting, there are two controllers involved: an expert 
controller, which performs the desired task with perfect knowledge of the 
environment, and a learned controller, which is trained to imitate the behaviour 
of the omniscient controller.

For each of the task that we are going to face, we introduce three controllers: the 
expert controller, that exploits its complete knowledge of the state of the system 
to decide the best action to perform, the manual controller, that has only a partial 
knowledge of it, and a controller learned by imitating the expert one.

These controllers will be described in detail later in the sections dedicated to each 
task.

\section{Approaches}
\label{sec:app}
\subsection{Distributed}
\label{subsec:dist}

\subsection{Distributed with communication}
\label{subsec:comm}