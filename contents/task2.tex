\section{Task 2}
\label{sec:task2}

\subsection{Overview}
\label{subsec:desc2}

The second scenario tackles another newsworthy multi-agent coordination 
task, 
colour according a certain criterion the robots in the space. 
The environment is the same described in Chapter \ref{chap:experiments} as 
well as all the assumption defined in Section \ref{subsec:desc1}.

The only difference lies in the goal shared by the robots: assuming that the agents 
are divided into groups, their purpose is to colour themselves, by turning on their 
top \gls{rgb} \gls{led}, depending on their membership group.

Initially, for the sake of simplicity, we decided to divide the robots into two 
groups. In particular, in case of an even number of agents, the robots in the first 
half of the row belong to the first group and the remaining to the second, while in 
the case of an odd number robots, the agent located in the central position is 
assigned to the first set.

As for the previous task, the problem can be solved performing imitation 
learning, instead, the role of communication is fundamental. In fact, what 
makes the difference are not the distances perceived by the robot sensors but the 
messages exchanged between the agents, that are of primal importance in order 
to determine their order. 

In this scenario, the two ``dead'' robots play an important role: they always 
communicate a message that indicates that they are the only two agents that 
receive communication just from one side.

Each agent can still update its state, that is the same as Task 1, by performing 
actions – in this case not moving along the x-axis but colouring the top \gls{led} 
in red or blue – based on the observations received from the environment – the 
messages received from neighbours – using their sensors. 

Then, by following the example of an expert, we train an end-to-end \gls{nn} 
that takes as input the communication – transmitted by the nearest neighbours in 
the previous timestep – and produces the colour of the robot together with the 
message to be sent.

\subsection{Controllers}
\label{subsec:task2controllers}

\subsubsection{Expert controller}
\label{subsubsec:omniscient2}

As disclosed in Section \ref{subsec:controllersmodel}, the first element involved in 
an imitation learning problem is an omniscient controller, also called expert, 
which is able to perceive the environment and the observation of all the agents, 
obtaining a global knowledge of the state of the system. In this way it can use all 
the available information that it owns to decide the best action to perform for all 
the agents.

In this scenario, the omniscient controller, based on the current poses of each 
robot, is able to define the order of the agents and turn on their top \gls{led}  in 
one 
time step.

Unfortunately, using an expert to solve this kind of problem is not realistic nor 
feasible in a real environment.

\subsubsection{Manual controller}
\label{subsubsec:manual2}

The second controller we want to write about is the manual one, whose main 
purpose is to draw conclusions about the quality of the controller learned.

The main difference between this and the previous controller is that, the manual 
one can be consider a local controller, that has only a partial knowledge of the 
environment since knows only the state of the current agent and its 
observations.

For this scenario, the observation of the robots, in particular the sensor readings, 
do not provide useful information about the order of the agents, therefore they 
are not used in order to accomplish this task.
On the other hand, if an omniscient controller is not employed, it is impossible to 
solve the problem without using communication, that is the only way for the 
agents to understand they ordering.

Thus, by initially making all robots transmit the same value, i.e. $0$, we are able 
to establish which are the first and the last robots in the row, or those that do not 
receive any communication respectively from the left and right. 
These two agents can at this point start the actual communication by transmitting 
the value received increased by $1$, i.e. $1$. 
The following robots will in turn transmit the value they have received increased 
by one. Since the messages received by each of them are two, the agents will in a 
sense, learn to count in order to understand which is the correct value to 
transmit.

The protocol used to decide the communication and the colour, which also 
depends on the amount of robots $N$, or if there are even or odd numbers, is 
shown in Listing \ref{lst:manualtask2}. The colour of each agent in the initial 
configuration is randomly chosen between the two possible colours, red and blue.

\begin{python}
c_left, c_right = get_received_communication(state)

if N % 2 == 1:  # if the number of robots is odd

	# Case 1: no communication received from left
	if c_left == 0:
		if c_right > N // 2:
			# the agent is in the first half of the row, so its colour is blue
			message = c_right - 1
			colour = 1
		elif c_right == N // 2:
			# the agent is the central one, so its colour is blue
			message = c_right + 1
			colour = 1
		else:
			# the agent is in the second half of the row, so its colour is red
			message = c_right + 1
			colour = 0
			
	# Case 2: no communication received from right
	elif c_right == 0:
		if c_left > N // 2:
			# the agent is in the second half of the row, so its colour is red
			message = c_left - 1
			colour = 0
		elif c_left == N // 2:
			# the agent is the central one, so its colour is blue
			message = c_left + 1
			colour = 1
		else:
			# the agent is in the first half of the row, so its colour is blue
			message = c_left+ 1
			colour = 1
			
	# Case 3: communication received from both sides
	else:
		if c_left > c_right:
			# the agent is in the second half of the row, so its colour is red
			message = c_right + 1
			colour = 0
		else:
			# the agent is in the first half of the row, so its colour is blue
			message = c_left + 1
			colour = 1


elif self.N % 2 == 0:  # if the number of robots is even

	# Case 1: no communication received from left
	if c_left == 0:
		if c_right > N // 2:
			# the agent is in the first half of the row, so its colour is blue
			# the situation is ambiguous the message to transmit could be c_right or 
			# even c_right - 1
			message = c_right
			colour = 1
		else:
			# the agent is in the second half of the row, so its colour is red
			message = c_right + 1
			colour = 0
	
	# Case 2: no communication received from right
	elif c_right == 0:
		if c_left < N // 2:
			# the agent is in the first half of the row, so its colour is blue
			message = c_left + 1
			colour = 1
		else:
			# the agent is in the second half of the row, so its colour is red
			# the situation is ambiguous the message to transmit could be c_left or 
			# even c_left - 1
			message = c_left
			colour = 0
	
	# Case 3: communication received from both sides
	else:
		if c_left > c_right:
			# the agent is in the second half of the row, so its colour is red
			message = c_right + 1
			colour = 0
		elif c_left < c_right:
			# the agent is in the first half of the row, so its colour is blue
			message = c_left + 1
			colour = 1
		else:
			# the agent is in the second half of the row, so its colour is red
			message = c_left
			colour = 0
\end{python}

\begin{lstlisting}[frame=none,caption=Protocol used from the manual controller 
to decide for each robot the message to transmit and the colour., 
label=lst:manualtask2]
\end{lstlisting}


\subsection{Communication approach}
\label{subsec:task2comm}

\subsubsection{Model training}
\label{subsubsec:learnedcomm2}
For this purpose, it is possible to use the same network used in the previous task, 
that means the one implemented for the distributed approach with 
communication, but this time ignoring the sensors readings.
Thus, using the same data collected before we build a model that at each time 
step takes as input for each robot only the message received in the previous time 
step, communicated by the nearest agents (on the left and on the right), and 
produces as output an array of 2 floats, the first one is the probability of the agent 
top \gls{led} to be blue and the second is the communication, i.e. the message 
transmitted by the robot to the nearest agents.

The structure of the communication network is shown in Figure 
\ref{fig:commnet2}, and as before it is composed by two nested 
modules: 
\begin{figure}[!htb]
	\centering
	\includegraphics[width=\textwidth]{contents/images/commnettask2}
	\caption[Communication network of the second task.]{Visualisation of the 
		forward pass of the communication network with three agents and a 
		sequence 
		composed by two time steps.}
	\label{fig:commnet2}
\end{figure}
in the high-level operates the \texttt{CommNet} that handle the 
sensing of all the agents, while in the low-level the \texttt{SingleNet} that 
works on the communication received by a single agent in a certain time 
step, producing as output the colour and the communication to transmit. 
The architecture of the \texttt{SingleNet}, displayed in Figure 
\ref{fig:singlenetcomm2}, is almost the same as the one of the distributed 
model without communication: there are three linear layers each of size 
$\langle \mathtt{input\_size}, 10\rangle$,  $\langle 10, 10\rangle$ and 
$\langle 10, 2\rangle$, where \texttt{input\_size} corresponds to the two 
communication values received, one from the left and one from the right.
\begin{figure}[!htb]
	\centering
	\includegraphics[width=.15\textwidth]{contents/images/task2allcomm}
	\caption[Network architectures for the communication approach.]{Visualisation 
	of the network architecture chosen for the 
		communication approach.}
	\label{fig:singlenetcomm2}
\end{figure}

The activation functions used for this purpose are two and are shown in 
Figure \ref{fig:activation}.
To the first and second layer is applied a Tanh non-linear activation 
function, while a sigmoid to the output, in order to normalise it in 
the range $[0, 1]$.
\begin{figure}[!htb]
	\begin{center}
		\begin{subfigure}[h]{0.495\textwidth}
			\includegraphics[width=\textwidth]{contents/images/sigmoid2}
			\caption{Tanh activation function}
		\end{subfigure}
		\hfill
		\begin{subfigure}[h]{0.495\textwidth}
			\includegraphics[width=\textwidth]{contents/images/tanh2}
			\caption{Sigmoid activation function.}
		\end{subfigure}
	\end{center}
	\caption{Trend of the activation functions applied as non-linear activations.}
	\label{fig:activation}
\end{figure}

As before, we use Adam optimiser but with a smaller learning rate, $0.01$. 
We split the dataset in mini-batches of size $10$ and then we train the models for 
$100$ epochs. 

In order to decide the metric to evaluate the goodness of the prediction it is 
necessary to analyse the output of the network. As we said, the model returns, in 
addition to the communication, the colour that is actually the probability of the 
agent top \gls{led} to be blue. This means that, when the network produces a 0, 
the probability that the \gls{led} is blue is equal to 0, i.e. is red, in the same way 1 
means that instead this will be blue. In this way, we define a simple policy function 
that returns the colour blue when the probability is between 0.5 and 1, and red 
otherwise, or when the probability is less than 0.5.
For this reason, instead of using the \gls{mse} loss function, as this is a binary 
classification problem we choose the \gls{bce}, defined as follows:
\begin{Equation}[H]
	\centering
	\begin{equation}
	\mathtt{BCE} = -\frac{1}{n} \sum_{i=1}^n \log(p(y_i)) + (1-y_i) \cdot \log(1 - 
	p(y_i))
	\end{equation}
	\caption{Binary Cross Entropy (\gls{bce}) loss function.}
	\label{eq:bce}
\end{Equation}
where $y$ is the target (1 for blue and 0 for red), $p(y)$ is the predicted 
probability of the \gls{led} of being blue for all $N$.
This loss function should return high values for bad predictions and low values for 
good ones.
It is important to remember that the communication is still learned in an 
unsupervised way.

\subsubsection{Experiments}
\label{subsubsec:expcomm2}
\subsubsection{Summary}
\label{subsubsec:summary1}

