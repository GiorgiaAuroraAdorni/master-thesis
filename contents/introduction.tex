\chapter{Introduction}
\label{chap:intro}
In this work we consider cooperative multi-agent scenarios, in which agents 
collaborate, and possibly communicate, to achieve a common goal.

Homogeneous Multi-Agent Systems (MAS) consider $N$ interacting agents, which 
have the same physical structure and observation capabilities, so they can be 
considered to be interchangeable and collaborate to solve a given task  
\cite[][]{stone2000multiagent, vsovsic2016inverse}.
This system is characterised by a state $S$ — which can be decomposed in sets of 
local states for each agent, the set of possible observations $O$ for each agent — 
obtained through sensors and the set of possible actions  $A$ for each agent.

The objective of this work is to use \gls{il} approaches to learn decentralised 
solutions via imitation of a centralised control, which indicates the action to be 
performed, by training end-to-end \glspl{nn}, either classifiers or regressors, that 
only exploit local observations and communications. This controller is the same 
for each agent, which means that given an identical set of observations as input, 
likewise, for each of these the outputs will be equivalent.

In this study, we focus on two different multi-agent scenarios: distributing the 
robots in space such that they stand at equal distance from each other, and, 
assuming that the agents are divided into two sets, colouring the robots in space 
depending on their group membership. 
For the sake of simplicity, in the second task, we decided to divide the robots into 
two groups: in case of an even number of agents, those in the first half of the row 
belong to the first group and the remaining to the second, while in the case of an 
odd number of robots, the same reasoning is applied and the central agent is 
assigned to the first set.
Both are examples of cooperative tasks based on the use of a distributed 
controller. While the second cannot be solved without allowing an explicit 
exchange of messages between the agents, in the first one, a communication 
protocol is not necessary, nonetheless, it may increase performance.

For the purpose of this work, we assumed that all the agents act in collaboration 
to achieve a common goal, except the first and last in the row that behave like 
walls. The two ``dead'' robots play an important role especially in the use of 
communication: they are the only ones who are able to immediately localise 
themselves since they can receive messages just from one side.
Moreover, we consider the Thymio as holonomic, since their movements are 
limited in only one dimension. This premise simplifies our system, in which 
consequently we have to keep into account only geometric constraints and not
kinematic.

To solve these tasks we have adopted two different methodologies, depending on 
whether the communication is used or not.
First of all, we analyse typical supervised learning approaches, which directly learn 
a mapping from observations to actions. This method can be applied only to the 
first task, in which a very simple ``distributed network'' that takes as input an 
array containing the response values of the sensors — which can be either 
\texttt{prox\_values},  \texttt{prox\_comm} or  \texttt{all\_sensors} — and 
produces as output an array containing one float that represents the speed of the 
wheels, which is assumed to be the same both right and left.
After that, we concentrate on more challenging situations where the 
communication is not provided to the network, instead, it is a latent variable 
which has to be inferred \cite[][]{le2017coordinated}.

Throughout the experiments, we demonstrate the effectiveness of our methods, 
comparing approaches with and without communication. We also analyse the 
effect of varying the inputs of the networks, the initial distance between the robot  
and the number of agents chosen.

\bigskip
\subsubsection*{Thesis Outline}
\label{subsec:outline}

This thesis is composed of 7 chapters, whose main points are presented as follows:
\begin{itemize}
	\item Chapter \ref{chap:stateoftheart} summarises the previous research 
	studies on the topic, evaluating the approaches adopted by the authors.
	
	\item Chapter \ref{chap:background} provides some background knowledge 
	needed to properly understand the research contents.
	
	\item Chapter \ref{chap:impl} presents the tools used for the data collection 
	and all the additional frameworks we relied on.
		
	\item Chapter \ref{chap:methods} thoroughly illustrates the methodology used, 
	their benefits and limitations, including also description of the kind of data 
	used and how they are collected. 

	\item Chapter \ref{chap:experiments} explores the analysis conducted and 
	shows evaluation results.	
	
	\item Chapter \ref{chap:concl} address the results of the experiments and 
	concludes the thesis by discussing the implication of our findings, possible fixes 
	and outlines future works.
	
\end{itemize}

