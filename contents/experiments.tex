\chapter{Experiments}
\label{chap:experiments}

%\section{Overview}
%\label{sec:overview}

In this work we investigate collaborative multi-agent problems in a \gls{1d} 
simulated environment.

The environment is represent by a Cartesian plane, a system of coordinates
defined by two perpendicular axis, $x$ and $y$. 

In this scenario, $N$ robots, all oriented in the same direction, are initially 
randomly placed along the x-axis, avoiding collisions and in such a way the 
average gap among the agents is included in the proximity sensors' ranges. 
All the robots act in collaboration to achieve a common goal, however the first 
and last in the row behave like walls. %% FIXME

The agent is a point in the plane, formally described by an homogeneous vector 
with respect to the world coordinate frame $W$, obtained multiplying the 
homogeneous vector of the point, with respect to the robot coordinate frame 
$A$, by an homogeneous transformation. 

The relative pose $A$ of each agent is identified by a $3 \times 3$ matrix 
$\mathbf{T}$, with respect to the world reference frame $W$. 

\begin{Equation}[!htb]
	\centering
	\begin{equation}
	\label{eq:homogeneous transformation matrix}
	{^W\!\xi_A} = {^W\!\mathbf{T}_A} 
	=
	\begin{pmatrix}
	^W\!\mathbf{R}_A & ^W\!\mathbf{t}_A\\
	0, 0 & 1
	\end{pmatrix}
	=
	\begin{pmatrix}
	\cos \theta & - \sin \theta & t_x\\
	\sin \theta & \cos \theta & t_y\\
	0 & 0 & 1
	\end{pmatrix}
	\end{equation}
	\caption[Homogeneous Transformation Matrix]{The homogeneous 
	transformation matrix, 	$^W\!\mathbf{T}_A$, includes $^W\!\mathbf{R}_A$, a 
	$2 \times 2$ rotation matrix and $^W\!\mathbf{t}_A$, a $2 \times 1$ 
	translation vector.}
	\label{eq:hommatrix}
\end{Equation}

However, since we are in a \gls{1d} environment, the $y$ coordinate is equal 
to $0$ and also the orientation angle $\theta$ must be zero as all the agents are 
oriented as the world frame. 

Moreover, we can consider the agents as holonomic, since their movements are 
limited in only one dimension. This premise simplifies our system, in which 
consequently we have to keep into account only geometric constraints and not
kinematic.

\section{Task 1}
\label{sec:task1}

\subsection{Overview}
\label{subsec:desc1}

The first scenario tackles an interesting multi-agent coordination task, the 
distribution of robots in space.

As described in the overview of Chapter \ref{chap:experiments}, in which are 
provided additional details, in a \gls{1d} environment are spawn in a ``single 
file'' N robots in random positions. %% FIXME within the proximity sensors range 
Each agent can update its state – its absolute position – by performing actions – 
moving forward and backwards along the x-axis – based on the observations 
received from the environment – the distances from neighbours – using their 
sensors.  They are also able to transmit and receive a communication value to 
peer robots within a range of about $48$\gls{cm}. 
A full explanation of how communication works for Thymio II is covered in 
Section \ref{subsec:thymiocomm}.

As a consequence of a \gls{1d} environment, the agents' movement are 
limited to two directions: moving towards the x-axis when the velocity is positive 
while going backwards when the speed is negative. 

The robots share a common goal: arrange themselves uniformly along the 
line between the two ``dead'' robots, in such a way they stand at equal distances 
from each other.

%% FIXME The problem presents a collaborative goal given the fact that the 
%%single 
%% agent can’t sense all the environment and the final configuration depends 
%%on 
%% the position of all of them.

This problem represents a cooperative goal that can be reached performing 
imitation learning, that means training an end-to-end \gls{nn} by following 
the example of an omniscient controller, introduced in Section 
\ref{subsubsec:omniscient}.
Exploiting its complete knowledge of the environment, % FIXME state of the 
%system 
the expert is then able to decide the best action to perform.

In the course of this study, we tried to understand if it is possible to use a 
controller learned by imitation, instead of using a manual one. In particular, we 
concentrate on two approaches, presented in Sections \ref{subsec:dist} and 
\ref{subsec:comm}.
In both cases, we train \glspl{dnn} that receive sensor inputs and produce 
commands for the motors, but for the second alternative, the network has an 
addition input – the received communication transmitted by the nearest agents in 
the previous timestep – and an extra output – the message to be sent.


%%FIXME
%In this work, we decided to define and implement two multi-agent scenarios in 
%which N agents have a common goal and that requires them to coordinate 
%themselves to achieve it. Besides being an excellent example of distributed tasks, 
%the two problems present an important difference. The first is possible to solve 
%without communication, that can help to reach a more efficient solution but it is 
%not strictly necessary. The second, instead, is impossible to solve without explicit 
%communication between the agents.

\subsection{Controllers}
\label{subsec:controllers}

\subsubsection{Expert controller}
\label{subsubsec:omniscient}

As disclosed in Section \ref{sec:controllers}, the first element involved in an 
imitation learning problem is an omniscient controller, also called expert.

%fixme frase poco chiara alla fine
This controller perceives the environment and the observation of all the agents, 
obtaining a global knowledge of the state of the system. In this way it can use all 
the available information that it owns to decide the best action to perform for all 
the agents. 

In this first scenario, the omniscient controller, based on the current poses of the 
robots, moves the agents at a certain speed to reach the target positions. In 
particular, the linear velocity of each agent is computer as a ``signed distance`` 
between the current and the goal position of the robot, along its theta. 

Formally, given the current pose, defined by the triple $(x, y, \theta)$ and the 
target pose $(\overline x, \overline y, \overline \theta)$, the signed distance $d$ 
is computed as follow:
\begin{Equation}[!htb]
	\centering
	\begin{equation}
	d = \left(\overline x * \cos (\theta) + \overline y * \sin (\theta)\right) -
	\left( x * \cos (\theta) + y * \sin (\theta)\right)
	\end{equation}
	\caption[Signed distance.]{Signed distance function.}
	\label{eq:signeddist}
\end{Equation}

\noindent
To obtain the final velocity of the agent, this quantity is multiplied by a constant, 
we choose $10$ to keep the controller as fast as possible, and then clipped to its 
maximum value.

This controller can be informally considered as a simple Bang Bang controller 
since the optimal controller move at maximum speed the robot towards the target 
unless the target is closer than %%FIXME è control step o control step duration?
\texttt{control\_step\_duration} $\times$ \texttt{maximum\_speed}. In this case, 
the agent is moved lower then the maximum allowed speed so that at the end of 
the timestep it is located exactly at the target.

Unfortunately, using an omniscient controller to solve this kind of problems is not 
realistic nor feasible in a real environment.

\subsubsection{Manual controller}
\label{subsubsec:manual}
%%FIXME
The second controller we want to discuss/write about is the manual one, whose 
main purpose is to draw conclusions about the quality of the controller learned.

The main difference between this and the previous controller is that, the manual 
one can be consider a local distributed controller, that has only a partial 
knowledge of the environment since knows only the state of the current agent 
and its observations.

This controller moves the robots towards the target by minimising the difference 
between the values recorded by the front and rear sensors, trying to maintain the 
maximum achievable speed.

For each agent the controller is the same and given the same set of observations 
as input the output will be the same.

The one implemented is a proportional controller, a particular variant of \gls{pid}, 
with only the $K_p$ term. 
The closed-loop control uses a feedback to adjust the control while the action 
takes place in proportion to the existing error. This function given a desired 
output $x(t)$, or set point, produces an output $y(t)$, or process variable, such 
that the error $e(t)$ is obtained as the difference between the value of the set 
point and the process variable. Finally, the control variable is the $u(t)$ is the 
output of the \gls{pid} controller and is computed as follows:

\begin{Equation}[!h]
	\centering
	\begin{equation}
	u(t) = K_p * e(t)
	\end{equation}
	\caption[Proportioal PID controller.]{Proportional \gls{pid} controller.}
	\label{eq:pid}
\end{Equation}

The value of the proportional gain has been tuned to yield satisfactory 
performance so that the system is stable, as shown in Figure \ref{fig:pid}. 
Moreover, since the value of the error is computed using the Equation 
\ref{eq:systemerror}, then $K_p$ should be positive.

\begin{Equation}[!h]
	\centering
	\begin{equation}
	e(t) = x(t) - y(t)
	\end{equation}
	\caption{Calculation of the error value $e(t)$ of the system.}
	\label{eq:systemerror}
\end{Equation}

\begin{figure}[htb]
	\centering
	\includegraphics[width=.5\textwidth]{contents/images/Step-responsep=kp5ki0kd0}
	\caption[Step response of the proportinal PID controller.]{Visualisation of the 
	step-response of a P controller with proportional 
	gain $5$.}
	\label{fig:pid}
\end{figure}

It is important to notice that the speed returned by the controller is used to set the 
\texttt{motor\_\{left, right\}\_target}, both with the same value in order to move 
the robots straight ahead. Moreover, the first and the last robots of the line, which 
are those which sensors never receive a response respectively from the back and 
from the front, never move.

%%FIXME
See \url{https://youtu.be/jNkt7xf6pUU} for a short video that shows the 
simulation of this task \cite[][]{task1manual}.

\subsection{Distributed approach experiments}
\label{subsec:ex1distr}
\subsubsection{Learned controller}
\label{subsubsec:learneddist}

Once, enough data are collected through the simulator by using the optimal 
controller, it is possible to train a very simplified ``distributed network'' 
that takes as input an array containing the response values of the sensors – 
which can be either \texttt{prox\_values}, \texttt{prox\_comm} or 
\texttt{all\_sensors} – and produces as output an array containing one float 
that represents the speed of the wheels, which is assumed to be the same 
both right and left.

The dataset then contains a fixed number of simulation runs, each of these 
composed by a variable quantity of timesteps. It is important to notice that 
for this approach, unlike the one with communication, it is not necessary to 
keep the order of the sequence of timesteps, neither to know the exact 
number of agents in the simulation since the network input is the sensing 
associated to a single robot.

For this reason, the model is independent of the number of agents and 
consequently it is possible to prove its generalisation capacity, regardless 
the number of robots, by training the networks first on datasets each with a 
different but fixed value of $N$ and then on simulations composed by a 
variable $N$.
It is easy to show that although the value of $N$ changes the network 
structure does not, as it is sufficient during the input preprocessing to 
change the dimension of the input in such a way that all the tensors have a 
the same length, fixed at the maximum possible value of $N$, padding 
those tensors with a lower number of agents.

Thanks to these two assumptions, it is possible to shuffle the original 
dataset, based on the single run, in order to improve the generalisation on 
the samples, and then split the resulting collection into the train, the 
validation and the test sets, containing respectively $60$-$20$-$20\%$ of 
the data. 

The architecture of the network, displayed in Figure 
\ref{fig:singlenetdistributed1}, is straightforward: there are three linear 
layers each of size $\langle\mathtt{input\_size}, 10\rangle$,  $\langle 10, 
10\rangle$ and $\langle 10, 1\rangle$, where \texttt{input\_size} is the 
shape of the sensing, that can be $7$ or $14$.

\begin{figure}[htb]
	\centering
	\begin{subfigure}[h]{0.495\textwidth}
		\centering
		\includegraphics[width=.3\textwidth]{contents/images/task1distributed@4x}%
		\caption{Structure of a network with $7$ input sensing.}
		\label{fig:singlenet7distributed1}
	\end{subfigure}
	\hfill
	\begin{subfigure}[h]{0.495\textwidth}
		\centering
		\includegraphics[width=.3\textwidth]{contents/images/task1distributed_all@4x}
		\caption{Structure of a network with $14$ input sensing.}
		\label{fig:singlenet14distributed1}
	\end{subfigure}
	\caption{Visualisation of the network architectures for the distributed 
		approach.}
	\label{fig:singlenetdistributed1}
\end{figure}

To the first and second layer is applied a non-linear activation function, 
useful to make the model generalise. 
In particular, we chose the hyperbolic tangent (Tanh) 
\cite[see][]{kalman1992tanh}, a zero-centred function, shown in Figure 
\ref{fig:tanh}, whose range lies between $[-1, 1]$ and its output is given by

\begin{Equation}[H]
	\centering
	\begin{equation}
	f(x)= \frac{\sinh (x)}{\cosh (x)} = \bigg( \frac{e^x - e^{-x}}{e^x + 
		e^{-x}}\bigg)
	\end{equation}
	\caption{Hyperbolic Tangent Function (Tanh).}
	\label{eq:tanh}
\end{Equation}

This type of activation function is often used in deep learning and one of is 
advantages is that negative inputs are mapped to strongly negative values.

\begin{figure}[htb]
	\centering
	\includegraphics[width=.5\textwidth]{contents/images/tanh2}%
	\caption{Trend of the Tanh activation function.}
	\label{fig:tanh}
\end{figure}

%fixme citation
As optimiser we chose Adam, {an algorithm for first-order gradient-based 
optimisation of stochastic objective functions, based on adaptive estimates of 
lower-order moments}, \cite[see][]{kingma2014adam, 
loshchilov2017decoupled}, 
implemented in the \texttt{torch.optim} package, with a learning rate of $0.01$. 

Instead of computing the gradient descent on the entire dataset, the training set is 
split in mini-batches of size $100$ and an approximation of the gradient is 
produced, which makes the algorithm faster and at the same time, for sufficiently 
large numbers, the result is indistinguishable.

Gradient descent algorithms are susceptible to ``get stuck'' in local minima.
Mini-batches shuffle facilitate to avoid this problem by enabling the gradient to 
``bounce'' out of eventual local minimum, making it more variable by exploiting 
randomness, thereby helping convergence.

All the models are trained for $50$ epochs and evaluated using the \gls{mse} loss 
function, often used in regression problems. 
This criterion, implemented in the \texttt{torch.nn} package, measures the 
average of squared error between predictions and targets and learns to reduce it 
by penalising big errors in the model predictions.

\begin{Equation}[H]
	\centering
	\begin{equation}
	\mathtt{MSE} = \frac{\sum_{i=1}^n (y_i-\hat y_i)^2}{n}
	\end{equation}
	\caption{Mean Squared Error (\gls{mse}) loss function.}
	\label{eq:mse}
\end{Equation}
	
\subsubsection{...}
\label{subsubsec:....}


\subsection{Distributed approach experiments with communication}
\label{subsec:ex1comm}


\subsubsection{Learned controller}
\label{subsubsec:learnedcomm}

Using the same dataset collected using the expert controller to perform the task 
without communication, it is possible to train a ``distributed network with 
communication'' that at each timestep takes as input for each robot an array 
containing the response values of the sensors – which can be either 
\texttt{prox\_values}, \texttt{prox\_comm} or \texttt{all\_sensors} – and the 
message received in the previous timestep, communicated by the nearest agents 
(one on the left and one on the right), and produces as output an array of 2 
floats, corresponding the first one to the control, that as before is the speed of the 
wheels, and the second one to the communication, i.e. the message transmitted 
by the robot to the nearest agents.

As before, the model is independent of the number of agents in the simulations, 
instead, in this approach is important to keep track of the timesteps order since 
the input of the network requires the communication received, that corresponds 
to a message transmitted in the previous timestep. 
To do so, a preprocessing is applied to the dataset in order to combine 
consecutive timesteps into a set of sequences. Therefore, we divide each 
simulation in sequences of length $2$, or composed by two consecutive 
timesteps, using a stride of $1$ among them, that contains an ordered series of 
two states for each robot.   

In this case, the input shape is transformed from $1 \times \mathtt{input\_size}$ 
to $\mathtt{seq\_length} \times N \times \mathtt{input\_size}$, where 
\texttt{seq\_length} is fixed at $2$, while $N$ is variable and \texttt{input\_size} 
can be $7$ or $14$.

\begin{figure}[htb]
	\centering
	\includegraphics[width=\textwidth]{contents/images/commnet}
	\caption[Communication network.]{Communication network.}
	\label{fig:commnet1}
\end{figure}

It is important to notice that the communication is not yet in the input since it is 
not contained in the dataset, instead is treated as a hidden variable to be inferred. 
In particular, since in the first timestep no messages have been received yet, we 
initialise randomly a placeholder, that has the dimension of the number of agents 
plus two, using float values in the range $[0, 1]$. 
Moreover, the communication value for the first and last agents would always be 
$0$, since we want to distinguish the fact that the two extreme robots never 
receive messages respectively from the left or from the right.
For this reason, we define a recurrent structure of the communication network, 
shown in Figure \ref{fig:commnet1}, composed by two nested modules: in the 
high-level operates the \texttt{CommNet} that handle the sensing of all the 
agents, while in the low-level \texttt{SingleNet} that works on the sensing and the 
communication received by a single agent in a certain timestep, producing as 
output the control and the communication to transmit. 

The architecture of the \texttt{SingleNet}, displayed in Figure 
\ref{fig:singlenetcomm1}, is almost the same as the one of the distributed model 
without communication: there are three linear layers each of size 
$\langle\mathtt{input\_size} + 2, 10\rangle$,  $\langle 10, 
10\rangle$ and $\langle 10, 2\rangle$, where \texttt{input\_size} is the sum of 
the shape of the sensing and the two communication values received, one from 
the left and one from the right.

\begin{figure}[H]
	\centering
	\begin{subfigure}[h]{0.495\textwidth}
		\centering
		\includegraphics[width=.3\textwidth]{contents/images/task1distributedcomm@4x}%
		\caption{Structure of the \texttt{SingleNet} with communication and $7$ 
		input sensing.}
	\end{subfigure}
	\hfill
	\begin{subfigure}[h]{0.495\textwidth}
		\centering
		\includegraphics[width=.3\textwidth]{contents/images/task1distributed_allcomm@4x}
		\caption{Structure of the \texttt{SingleNet} with communication and $14$ 
		input sensing.}
		\label{fig:singlenet14comm1}
	\end{subfigure}
	\caption{Visualisation of the network architectures for the distributed 
		approach with communication.}
	\label{fig:singlenetcomm1}
\end{figure}

As before, to the first and second layer is applied a tanh non-linear activation 
function, while a sigmoid \cite[see][]{han1995influence}, shown in Figure 
\ref{fig:sigmoid}, is applied to the second dimension of the output, that is the 
value of the communication to transmit, in order to normalise it in the range $[0, 
1]$ and its output is given by
\begin{Equation}[htb]
	\centering
	\begin{equation}
	\sigma(x)= \frac{1}{1 + e - x}
	\end{equation}
	\caption{Sigmoid Function.}
	\label{eq:sigmoid}
\end{Equation}

\noindent
Since applying this activation function the message to be transmitted is a float 
between $0$ and $1$, it will be later transformed into an integer and rescaled 
between $0$ and $1023$ so that it is in the format expected by the Thymio.

\begin{figure}[htb]
	\centering
	\includegraphics[width=.5\textwidth]{contents/images/sigmoid2}%
	\caption{Trend of the Sigmoid activation function.}
	\label{fig:sigmoid}
\end{figure}

As before, we use Adam optimiser but with a smaller learning rate, $0.001$. We 
split the dataset in mini-batches, this time of size $10$ and then train the models 
for $500$ epochs. 

Finally we evaluate the goodness of the predicted control using the \gls{mse} loss 
function, while the communication is learned in an unsupervised way.
Since the network is fully connected, the communication affects directly the 
output, and consequently, the error minimised, even if it is computed using only 
the control. Improving the loss have an impact also on the communication latent 
variable: since the error is propagated through the internal network, in order to 
update the weight during the back-propagation step, that affects the 
communication.

\subsection{Results}
\label{subsec:results1}


\section{Task 2: Colouring the robots in space}
\label{sec:task2}

The second scenario tackles another multi-agent coordination task, assuming that 
the agents are divided into groups, their purpose is to colour themselves, by 
turning on their top \gls{rgb} \gls{led}, depending on their group membership. 
As for the previous task, the problem can be solved performing imitation 
learning, but the role of communication is fundamental. In fact, what makes the 
difference are not the distances perceived by the robot sensors but the messages 
exchanged between the agents, which are they only mean to determine their 
order. 
In this scenario, the two ``dead'' robots play an important role: they always 
communicate a message that indicates that they are the only two agents that 
receive communication just from one side.

\subsection{Distributed approach with communication}
\label{subsec:task2-exp-comm}

\subsubsection{Experiment 1: variable number of agents}
\label{subsubsec:task2-exp-comm-1}
In this section, we explore the experiments carried out using the communication 
approach, in particular, examining the behaviour of the control learned from 9 
networks 
\begin{figure}[!htb]
	\centering
	\begin{tabular}{ccc}
		\toprule
		\textbf{Model} \quad & \textbf{\texttt{avg\_gap}} & \textbf{\texttt{N}}\\
		\midrule
		\texttt{net-v1}   &  $8$		 &	 $5$ \\
		\texttt{net-v2}   &  $20$		&	$5$ \\
		\texttt{net-v3}   &  variable   &    $5$\\
		\texttt{net-v4}   &  $8$		 &	  $8$ \\
		\texttt{net-v5}   & $20$   		&	 $8$ \\
		\texttt{net-v6}   &  variable	&	 $8$ \\
		\texttt{net-v7}   &  $ 8$		  &	 variable\\
		\texttt{net-v8}   &  $20$		 &	variable\\
		\texttt{net-v9}   &  variable	 &	variable\\
		\bottomrule
	\end{tabular}
	\captionof{table}[Experiments with variable agents and gaps 
	(communication).]{List of the experiments carried out using a variable number 
		of agents and of gaps.}
	\label{tab:modelcommt2}
\end{figure}

\noindent
based on different simulation runs that use a number of robots $N$ that 
can be fixed at $5$ or $8$ for the entire simulation, or even vary in the range $[5, 
10]$, and an \texttt{avg\_gap} that can be a fixed value in all the runs, chosen 
between $8$ or $20$, but also vary in the range $[5, 24]$. 
The objective of this set of experiments, summarised in Table 
\ref{tab:modelcommt2}, is to verify the robustness of the communication 
protocol and prove also the scalability of the network on the number of agents.

First of all, we show in Figure \ref{fig:t2lossallt} an overview of the train and 
validation losses obtained for these models.
\begin{figure}[!htb]
	\centering
	\includegraphics[width=.8\textwidth]{contents/images/task2/loss-communication-all@}%
	\caption[Comparison of losses of the second set of experiments.]{Comparison 
		of the losses of the models carried out using a variable number of agents and 
		of average gap.}
	\label{fig:t2lossallt}
\end{figure}

\paragraph*{Results using 5 agents}

We start our examination by inspecting the behaviour of the network trained on 
simulations with variable average gap, i.e., \texttt{net-v3}, \texttt{net-v6} and 
\begin{figure}[!htb]
	\centering
	\includegraphics[width=.45\textwidth]{contents/images/task2/loss-communication-N5}
	\caption[Comparison of the losses of the models that use $5$ 
	agents.]{Comparison of the losses of the models that use $5$ agents as 
		the gap varies.}
	\label{fig:commlossn5t2}
\end{figure}

\noindent
\texttt{net-v9}, summarising, in Figure \ref{fig:commlossn5t2}, the losses of 
these experiments in order to highlight the difference of performance using a gap 
that is first small, then large and finally variable, respectively represented by the 
blue, the orange and the green lines.
Clearly, in case of small gaps the network performs better, albeit slightly, as the 
agents are already close to the target.

Then, we move to explore the results of the experiments by showing in Figure 
\ref{fig:net-v3auc} the \gls{roc} curve of the model 
\cite[][]{fawcett2006introduction}, a visualisation of the performance of our 
classification model, in terms of \gls{tpr} versus \gls{fpr}, at all classification 
thresholds.
In particular we use the \gls{auc} to evaluate the classifier: by measuring the 
\gls{2d} area under the ROC curve, from $[0, 0]$ to $[1, 1]$, the \gls{auc} is able 
to provide an aggregate measure of performance as the discrimination threshold 
varies.
We assume that a model whose predictions are 100\% correct has an \gls{auc} of 
1, as in this case.
\begin{figure}[!htb]
	\centering
	\includegraphics[width=.5\textwidth]{contents/images/net-v3/roc-net-v3(a)}%
	\caption[Evaluation of the ROC of \texttt{net-v3}.]{Visualisation of the 
		\gls{roc} curve of \texttt{net-v3}.}
	\label{fig:net-v3auc}
\end{figure}

Also for this task, it is interesting to analyse the type of communication protocol 
inferred by the network, also comparing it with the one implemented by the 
manual controller. 
In Figure \ref{fig:net-v3commcolour} are shown, for a simulation run, first the 
messages transmitted by the agents over time, through a colour bar whose 
spectrum is included in the range [0, 1], i.e.,the maximum and minimum value of 
communication transmitted, and then the colour assumed by the robot in a 
certain time step, for both the manual and the learned controllers.
The extreme robots always transmit the same message using both controllers, 
while using the learned one, the central robots seems to transmit the same value, 
i.e.,1, but despite this they are able to achieve their goal in only two time steps, 
one less than with the manual. This behaviour cannot scale to a number of robot 
higher then $5$. For instance, in case of $5$ agents, in the first time step, 
\texttt{myt2} and \texttt{myt4} receive respectively the messages $(0, 1)$ and $(1, 
0)$, so they immediately know their position with respect to the central robot, 
which in turn knows its position since it receives $(1, 1)$. Then they communicate 
their message and in the following time step all the agents have coloured 
themselves in the right way, achieving the goal.
Consequently if the number of robots is greater, the central robots are not able to 
localise themselves. 
\begin{figure}[!htb]
	\begin{subfigure}[h]{\textwidth}
		\centering
		\includegraphics[width=.6\textwidth]{contents/images/net-v3/net-v3-manual-0(1)}
		\caption{Communication and colour decided using the manual controller.}
	\end{subfigure}
	\hspace*{\fill}%          % empty line absolutely necessary!
	\vspace*{8pt}%  
	\hspace*{\fill}%  
	\begin{subfigure}[h]{\textwidth}
		\centering			
		\includegraphics[width=.6\textwidth]{contents/images/net-v3/net-v3-learned-0(1)}
		\caption{Communication and colour decided using the learned controller.}
	\end{subfigure}
	\caption[Evaluation of the communication learned by 
	\texttt{net-v3}.]{Visualisation of the communication transmitted by each 
		robot over time and the colour decided by the controller learned from 
		\texttt{net-v3}.}	
	\label{fig:net-v3commcolour}
\end{figure}
\vspace{0.5cm}

In Figure \ref{fig:net-v3error} is presented a useful metric that measures the 
amount of wrong expected colours, on the y-axis, over time, averaged for all the 
robots among the simulation runs. In particular, at each time step we count the 
number of agents that have the wrong colour and divide it by the number of 
simulations.
The mean value is shown as well as the bands representing minus and plus the 
standard deviation.
On average, the amount of correct colours is higher for the manual controller 
than the learned one. 
\begin{figure}[!htb]
	\centering
	\includegraphics[width=.5\textwidth]{contents/images/net-v3/colours-errors-compressed}%
	\caption[Evaluation of \texttt{net-v3} amount of wrong expected 
	colours.]{Comparison of performance in terms of amount of wrong expected 
		colours obtained using the controller learned from \texttt{net-v3}.}
	\label{fig:net-v3error}
	\vspace{-0.5cm}
\end{figure}

\paragraph*{Results using 8 agents}
Following are presented the results of the experiments performed using $8$ 
agents. 
\begin{figure}[H]
	\centering
	\includegraphics[width=.5\textwidth]{contents/images/task2/loss-communication-N8}
	\caption[Comparison of the losses of the models that use $8$ 
	agents.]{Comparison of the losses of the models that use $8$ agents as 
	the gap varies.}
	\label{fig:commlossn8t2}
\end{figure}

\noindent
In Figure \ref{fig:commlossn8t2} are summarised the performance in terms of 
train and validation losses, by varying the average gap, as before the blue, orange 
and green lines represent respectively gaps of $8$cm, $20$cm and 
variable. 
From a first observation we see that the losses are higher than before, this is 
because a great number of agents reduce the performance, since more time steps 
are necessary to achieve the goal.

From the \gls{roc} curve of the model in Figure \ref{fig:net-v6auc} we observe 
that this time the \gls{auc} is decreased from 1 to 0.87 with respect to the 
previous model examined. 
%, while the accuracy is increased from $66\%$ up to $73\%$, 
\begin{figure}[!htb]
	\centering
	\includegraphics[width=.5\textwidth]{contents/images/net-v6/roc-net-v6(a)}%
	\caption[Evaluation of the \gls{roc} of \texttt{net-v6}.]{Visualisation of the 
		\gls{roc} curve of \texttt{net-v6} based on \gls{bce} Loss.}
	\label{fig:net-v6auc}
\end{figure}

\bigskip
In Figure \ref{fig:net-v6error} is presented the measure of the amount of wrong 
expected colours, on the y-axis, over time, averaged on all robots among all the 
simulation runs. 
\begin{figure}[!htb]
	\centering
	\includegraphics[width=.5\textwidth]{contents/images/net-v6/colours-errors-compressed}%
	\caption[Evaluation of \texttt{net-v6} amount of wrong expected 
	colours.]{Comparison of performance in terms of amount of wrong expected 
		colours obtained using the controller learned from \texttt{net-v6}.}
	\label{fig:net-v6error}
\end{figure}

\noindent
As expected, the number of colours correctly predicted by the learned controller 
is lower than before, while the manual controller still has the same performance.  

Finally, we visualise, in Figure \ref{fig:net-v6commcolour}, the communication 
protocol inferred by the network and the one chosen by the manual controller, as 
well as the colour assumed by each robot, we immediately see a difference in both 
the figures.
This time it is possible to hypothesize the policy adopted by the network to send 
messages: as before, the extreme robots always send the same message but this 
time the central ones communicate a value interpreted as a reward. In detail, 
starting from the edges, the value 0 is transmitted, then, once the robot that 
follows or precedes receives this value in turn communicates 0, until all the agents 
have received the message and therefore have clear their positional order.
This type of reward acts in such a way as to colour as desired the following robot, 
for those in the first half, or the one that precedes, for those in the second, 
communicating 0 respectively when there is a red agent behind it or when in front 
there is a blue one. In this way the control is able to stabilise and achieve its goal.
\begin{figure}[!htb]
	\begin{subfigure}[h]{\textwidth}
		\centering
		\includegraphics[width=.55\textwidth]{contents/images/net-v6/net-v6-manual-0}
		\caption{Communication and colour decided using the manual controller.}
	\end{subfigure}
	\hspace*{\fill}%          % empty line absolutely necessary!
	\vspace*{8pt}%  
	\hspace*{\fill}%  
	\begin{subfigure}[h]{\textwidth}
		\centering			
		\includegraphics[width=.55\textwidth]{contents/images/net-v6/net-v6-learned-0}
		\caption{Communication and colour decided using the learned controller.}
	\end{subfigure}
	\caption[Evaluation of the communication learned by 
	\texttt{net-v6}.]{Visualisation of the communication transmitted by each 
		robot over time and the colour decided by the controller learned from 
		\texttt{net-v6}.}	
	\label{fig:net-v6commcolour}	
	\vspace{-0.5cm}
\end{figure}

\paragraph*{Results using variable agents}
We conclude the experiment on this task by presenting the results obtained using 
variable number of agents. In Figure \ref{fig:commlossnvart2} are summarised the 
performance in terms of loss, as before we used blue, orange and green lines to 
represent respectively average gaps of $8$cm, $13$cm and variable. 
As before we observe that in general the losses are increased respect the first 
experiment that use a smaller number of agents, since a higher amount of robots 
reduce the performance of the models, instead it is decreased respect the last 
experiment examined.
\begin{figure}[!htb]
	\centering
	\includegraphics[width=.47\textwidth]{contents/images/task2/loss-communication-Nvar}
	\caption[Comparison of the losses of the models that use variable 
	agents.]{Comparison of the losses of the models that use variable agents 
	as the gap varies.}
	\label{fig:commlossnvart2}
\end{figure}

From the \gls{roc} curve of the model in Figure \ref{fig:net-v9auc}  we observe 
that this time the \gls{auc} is a bit increased respect the previous experiment, 
going from $0.87$ up to $0.89$, but still worse than the first one.
\begin{figure}[!htb]
	\centering
	\includegraphics[width=.47\textwidth]{contents/images/net-v9/roc-net-v9(a)}%
	\caption[Evaluation of the \gls{roc} of \texttt{net-v9}.]{Visualisation of the 
		\gls{roc} curve of \texttt{net-v9}.}
	\label{fig:net-v9auc}
\end{figure}

This time we visualise in Figures \ref{fig:net-v9commcolour} and 
\ref{fig:net-v9commcolour2} two examples of communication protocol inferred 
by the network and the one chosen by the manual controller, as well as the colour 
assumed by each robot.
The first visualisation is obtained from a simulation with 10 agents. 
As before, the policy adopted by the network to send messages is very similar to 
the previous one. 
\begin{figure}[!htb]
	\begin{subfigure}[h]{\textwidth}
		\centering
		\includegraphics[width=.6\textwidth]{contents/images/net-v9/net-v9-manual-0}
		\caption{Communication and colour decided using the manual controller.}
	\end{subfigure}
	\hspace*{\fill}%          % empty line absolutely necessary!
	\vspace*{8pt}%  
	\hspace*{\fill}%  
	\begin{subfigure}[h]{\textwidth}
		\centering			
		\includegraphics[width=.6\textwidth]{contents/images/net-v9/net-v9-learned-0}
		\caption{Communication and colour decided using the learned controller.}
	\end{subfigure}
	\caption[Evaluation of the communication learned by 
	\texttt{net-v9}.]{Visualisation of the communication transmitted by each 
		robot over time and the colour decided by the controller learned from 
		\texttt{net-v9}.}	
	\label{fig:net-v9commcolour}
\end{figure}
The robots at the edges start to transmit the value 0. Then, once 
the next robots receive the message in turn they communicates 0 or a value very 
close to it, until all the agents have received and sent the the message and have 
finally clear their positional order.
This time the manual and learned controllers achieve the goal in the same 
number of time steps.
The second visualisation is obtained from a simulation with 6 agents. 
The policy adopted by the network is the same as before, this time is even more 
efficient than the protocol used from the manual controller. In fact, in 2 time 
steps, one less then the other controller, the model is able to achieve the goal.

Finally, in Figure \ref{fig:net-v9error} is presented the measure of the amount of 
wrong expected colours, on the y-axis, over time, averaged on all robots among 
all the simulation runs. 

\begin{figure}[!htb]
	\begin{subfigure}[h]{\textwidth}
		\centering
		\includegraphics[width=.5\textwidth]{contents/images/net-v9/net-v9-manual-1}
		\caption{Communication and colour decided using the manual controller.}
	\end{subfigure}
	\hspace*{\fill}%          % empty line absolutely necessary!
	\vspace*{8pt}%  
	\hspace*{\fill}%  
	\begin{subfigure}[h]{\textwidth}
		\centering			
		\includegraphics[width=.5\textwidth]{contents/images/net-v9/net-v9-learned-1}
		\caption{Communication and colour decided using the learned controller.}
	\end{subfigure}
	\caption[Evaluation of the communication learned by 
	\texttt{net-v9}.]{Visualisation of the communication transmitted by each 
		robot over time and the colour decided by the controller learned from 
		\texttt{net-v9}.}	
	\label{fig:net-v9commcolour2}
\end{figure}

\begin{figure}[H]
\centering
\includegraphics[width=.5\textwidth]{contents/images/net-v9/colours-errors-compressed}%
\caption[Evaluation of \texttt{net-v9} amount of wrong expected 
colours.]{Comparison of performance in terms of amount of wrong expected 
	colours obtained using the controller learned from \texttt{net-v9}.}
\label{fig:net-v9error}
\end{figure}

\noindent
for this experiment the number of colours correctly predicted by the learned 
controller is a bit less than that obtained by the manual controller, and even if the 
variance for the model is higher the performance are acceptable.

\paragraph*{Summary}
To sum up, we show the losses of the trained models as the number of robots vary 
for each gap, in particular, in blue, orange and green we refer to the simulation 
with $5$, $8$ and variable agents. 
Unlike the previous task, here no clear differences are highlighted varying the 
gap. 
The performance obtained with a smaller number of agents are clearly superior, 
while the other two, obtained by increasing the amount of robots, are very similar 
and tend to move away from each other as the gap grows.
\begin{figure}[!htb]
	\begin{center}
		\begin{subfigure}[h]{0.32\textwidth}
			\includegraphics[width=\textwidth]{contents/images/task2/loss-communication-gap_8}%
			\caption{\texttt{avg\_gap} of $8$cm.}
		\end{subfigure}
		\hfill
		\begin{subfigure}[h]{0.32\textwidth}
			\includegraphics[width=\textwidth]{contents/images/task2/loss-communication-gap_20}%
			\caption{\texttt{avg\_gap} of $20$cm.}
		\end{subfigure}
		\hfill
		\begin{subfigure}[h]{0.32\textwidth}
			\includegraphics[width=\textwidth]{contents/images/task2/loss-communication-gap_var}
			\caption{\texttt{avg\_gap} variable.}
		\end{subfigure}
	\end{center}
	\vspace{-0.5cm}
	\caption[Losses summary of the second task 
	(communication).]{Comparison of the losses of the model trained with 
	communication, by varying the number of agents for the three gaps.}
	\label{fig:commlosst2}
	\vspace{-0.5cm}
\end{figure}

\subsubsection{Experiment 2: increasing number of agents}
\label{subsubsec:task2-exp-comm-2}

The last group of experiments focuses on the scalability properties of a 
multi-agent system, showing the behaviour of the network trained using variable 
gaps and number of agents, applied on simulations with a higher number of 
robots, from 5 up to 50.

In Figure \ref{fig:errorcomm} is visualised, for 5 different experiments, the 
expected percentage of wrong colours over time, averaged for all the robots 
among the simulation runs. 
In all experiments, the number of errors in the simulation corresponds to 50\%, 
i.e.,half of the colours are wrong. This results are expected since the colours at the 
first time step are chosen randomly. As the time steps pass, the number of errors 
decreases at a constant speed, about two robots per time step. In general, 
$\frac{N}{2}$ time steps are required to achieve convergence, sometimes 
$\frac{N}{2} - 1$ when the number of agents is odd. 
Despite this, when using a number of robots between 5 and 10, on average 1\% of 
the colours are wrong in the final configuration. Increasing the amount of agents 
this value increases, reaching 10\% in the case of 50 robots.
Despite this, the performances are very promising and the network is able to scale 
well by increasing the number of robots.
\begin{figure}[!htb]
	\centering
	\includegraphics[width=.5\textwidth]{contents/images/colours-errors-compressed}%
	\caption[Evaluation of distances from goal for a high number of 
	robots.]{Comparison of performance in terms of distances from goal obtained 
		on simulations with an increasing number of robots.}
	\label{fig:errorcomm}
\end{figure}

\subsubsection{Remarks}
\label{subsubsec:remarks-task2-comm}

In this section we have shown that, for some problems, communication is a 
necessity. 
Therefore, using a controller learned through imitation learning, which 
autonomously infers a communication protocol between the agents, it is possible 
to obtain excellent results and solve the task in many cases more effectively than 
the baseline.
Moreover, the network is able to scale with the increase of the number of robots, 
without worsening performance.

