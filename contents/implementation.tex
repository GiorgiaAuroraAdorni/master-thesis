\chapter{Implementation and tools}
\label{chap:impl}

\section{Thymio II}
\label{sec:thymio}

The target platform chosen is Thymio II, a small differential drive mobile robot 
developed in the context of a collaboration between the MOBOTS group of the 
\gls{epfl} and the \gls{ecal}. 

Thymio runs the Aseba open-source programming environment 
\cite[see][]{magnenat2010aseba}, an event-based modular architecture for 
distributed control of mobile robots, designed to enable beginners to program 
easily and efficiently \cite[][]{mondada2017bringing}, making it well-suited for 
robotic education and research.

Another particularity of this tool is the integration with the open-source 
\gls{ros} \cite[][]{quigley2009ros}, through asebaros bridge \cite[][]{asebaros}. 

The Thymio II includes sensors that can measure light, sound and distance. It can 
perform actions such as move using two wheels, each powered by its own motor, 
but also turning lights on and off.

\subsection{Motors}
\label{subsection: motors}
The robot is equipped with two motors, each connected to one of the two 
wheels, which allow the robot to move forward, backward but also turn by setting 
the velocity of the wheels at different speeds. The maximum speed allowed to the 
agent is $16.6$ \gls{cm/s}.
