In recent years, robotics research has dedicated an extensive attention to 
cooperative multi-agents problems, in which agents have a common goal that 
requires them to collaborate, and possibly to communicate, to achieve it 
\cite[][]{ismail2018survey}.
Imitation learning algorithms successfully address the issue of coordinating robot 
swarms, providing new insights and promising solutions. 
These methods learn a controller by observing demonstrations of an expert, in 
other words the behaviour of a centralised controller, which can perceive the 
entire environment, including the state and observations of all agents 
\cite[][]{billard2008survey, zhang2018deep}. 

Performing tasks with complete knowledge of the state of a system is relatively 
easy, but centralised solutions might not be feasible in real scenarios since agents 
do not have direct access to the state but only to their own observations.
To overcome this issue, we train end-to-end \glspl{nn}, usually classifiers or 
regressors, that only exploit local observations and communications, learning 
decentralised solution via imitation of a centralised control, which indicates the 
action to be performed \cite[][]{ross2011reduction, tolstaya2020learning}.

In this study, we focus on two different scenarios: distributing the robots in space 
such that they stand at equal distance from each other, and colouring the robots 
in space depending on their group membership.
Both are examples of cooperative tasks based on the use of a distributed 
controller. While the second cannot be solved without allowing an explicit 
exchange of messages between the agents, in the first one, a communication 
protocol is not necessary, although it may increase performance.

The experiments are run in Enki \cite[][]{enki}, a high-performance open-source 
simulator for planar robots, which provides collision detection and limited physics 
support for robots evolving on a flat surface. Moreover, it can simulate groups of 
robots hundreds of times faster than real-time.

In addition to analysing typical supervised learning approaches, which directly 
learn a mapping from observations to actions, we concentrate on more 
challenging situations where the communication is not provided to the network, 
instead it is a latent variable which has to be inferred \cite[][]{le2017coordinated, 
stadie2017third}.

The results show how applying a communication strategy improves the 
performance of the distributed model, letting it decide which actions to take 
almost as precisely and quickly as the expert controller.